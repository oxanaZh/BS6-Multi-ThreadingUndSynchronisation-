% Kommentare für den Editor (TexWorks/TexMakerX)
% !TeX encoding   = utf8
% !TeX spellcheck = de-DE

% Dokumentenklasse (Koma Script) -----------------------------------------
\documentclass[%
   %draft,     % Entwurfsstadium
   final,      % fertiges Dokument
   paper=a4, paper=portrait, pagesize=auto, % Papier Einstellungen
   fontsize=11pt, % Schriftgröße
   ngerman, % Sprache
 ]{scrartcl} % Classes: scrartcl, scrreprt, scrbook

% ~~~~~~~~~~~~~~~~~~~~~~~~~~~~~~~~~~~~~~~~~~~~~~~~~~~~~~~~~~~~~~~~~~~~~~~~
% encoding
% ~~~~~~~~~~~~~~~~~~~~~~~~~~~~~~~~~~~~~~~~~~~~~~~~~~~~~~~~~~~~~~~~~~~~~~~~

% Encoding der Dateien (sonst funktionieren Umlaute nicht)
\usepackage[utf8]{inputenc}

% Encoding der Verzeichnisse (für Pfade mit Umlauten und Leerzeichne)
\usepackage[%
   extendedchars, encoding, multidot, space,
   filenameencoding=latin1, % Windows XP, Vista, 7
   % filenameencoding=utf8,   % Linux, OS X
]{grffile}

% ~~~~~~~~~~~~~~~~~~~~~~~~~~~~~~~~~~~~~~~~~~~~~~~~~~~~~~~~~~~~~~~~~~~~~~~~
% Pakete und Stile
% ~~~~~~~~~~~~~~~~~~~~~~~~~~~~~~~~~~~~~~~~~~~~~~~~~~~~~~~~~~~~~~~~~~~~~~~~
% Schriften
% ~~~~~~~~~~~~~~~~~~~~~~~~~~~~~~~~~~~~~~~~~~~~~~~~~~~~~~~~~~~~~~~~~~~~~~~~
% Fonts Fonts Fonts
% ~~~~~~~~~~~~~~~~~~~~~~~~~~~~~~~~~~~~~~~~~~~~~~~~~~~~~~~~~~~~~~~~~~~~~~~~

% immer laden:
\usepackage[T1]{fontenc} % T1 Schrift Encoding
\usepackage{textcomp}	 % Zusätzliche Symbole (Text Companion font extension)

% ~~~~~~~~~~~~~~~~~~~~~~~~~~~~~~~~~~~~~~~~~~~~~~~~~~~~~~~~~~~~~~~~~~~~~~~~
% Symbole
% ~~~~~~~~~~~~~~~~~~~~~~~~~~~~~~~~~~~~~~~~~~~~~~~~~~~~~~~~~~~~~~~~~~~~~~~~

\usepackage{amssymb}
\usepackage{mathcomp}


%% ==== Zusammengesetzte Schriften  (Sans + Serif) =======================

%% - Latin Modern
\usepackage{lmodern}
%% -------------------

%% - Bera Schriften
%\usepackage{bera}
%% -------------------

%% - Times, Helvetica, Courier (Word Standard...)
%\usepackage{mathptmx}
%\usepackage[scaled=.90]{helvet}
%\usepackage{courier}
%% -------------------

%% - Palantino , Helvetica, Courier
%\usepackage{mathpazo}
%\usepackage[scaled=.95]{helvet}
%\usepackage{courier}
%% -------------------

%% - Charter, Bera Sans
%\usepackage{charter}\linespread{1.05}
%\renewcommand{\sfdefault}{fvs}
%\usepackage[charter]{mathdesign}



%%%% =========== Typewriter =============

%\usepackage{courier}                   %% --- Courier
%\renewcommand{\ttdefault}{cmtl}        %% --- CmBright Typewriter Font
%\usepackage[%                          %% --- Luxi Mono (Typewriter)
%   scaled=0.9
%]{luximono}



% Pakete Laden
% ~~~~~~~~~~~~~~~~~~~~~~~~~~~~~~~~~~~~~~~~~~~~~~~~~~~~~~~~~~~~~~~~~~~~~~~~
% These packages must be loaded before all others
% (primarily because they are required by other packages)
% ~~~~~~~~~~~~~~~~~~~~~~~~~~~~~~~~~~~~~~~~~~~~~~~~~~~~~~~~~~~~~~~~~~~~~~~~
\usepackage{calc}
\usepackage{fixltx2e}	% Fix known LaTeX2e bugs

\usepackage[ngerman]{babel} 			% Sprache
\usepackage[dvipsnames, table]{xcolor} 	% Farben

% ~~~~~~~~~~~~~~~~~~~~~~~~~~~~~~~~~~~~~~~~~~~~~~~~~~~~~~~~~~~~~~~~~~~~~~~~
% Bilder, Gleitumgebungen und Platzierung
% ~~~~~~~~~~~~~~~~~~~~~~~~~~~~~~~~~~~~~~~~~~~~~~~~~~~~~~~~~~~~~~~~~~~~~~~~

\usepackage[]{graphicx}					% Graphiken
\usepackage{epstopdf}		% konvertiert eps in pdf

% provides new floats and enables H float modifier option
\usepackage{float}
% Floats immer erst nach der Referenz setzen
\usepackage{flafter}
% Alel Floats werden vor der nächsten section ausgegeben
\usepackage[section]{placeins} 
%

% ~~~~~~~~~~~~~~~~~~~~~~~~~~~~~~~~~~~~~~~~~~~~~~~~~~~~~~~~~~~~~~~~~~~~~~~~
% Beschriftungen (captions)
% ~~~~~~~~~~~~~~~~~~~~~~~~~~~~~~~~~~~~~~~~~~~~~~~~~~~~~~~~~~~~~~~~~~~~~~~~

\usepackage{caption}
\usepackage{subcaption}

% ~~~~~~~~~~~~~~~~~~~~~~~~~~~~~~~~~~~~~~~~~~~~~~~~~~~~~~~~~~~~~~~~~~~~~~~~
% Math
% ~~~~~~~~~~~~~~~~~~~~~~~~~~~~~~~~~~~~~~~~~~~~~~~~~~~~~~~~~~~~~~~~~~~~~~~~

% Base Math Package
\usepackage[fleqn]{amsmath} 
% Warnt bei Benutzung von Befehlen die mit amsmath inkompatibel sind.
\usepackage[all, error]{onlyamsmath}

% ~~~~~~~~~~~~~~~~~~~~~~~~~~~~~~~~~~~~~~~~~~~~~~~~~~~~~~~~~~~~~~~~~~~~~~~~
% Science
% ~~~~~~~~~~~~~~~~~~~~~~~~~~~~~~~~~~~~~~~~~~~~~~~~~~~~~~~~~~~~~~~~~~~~~~~~

% Einheiten und Zahlenformatierung
\usepackage{siunitx}

% ~~~~~~~~~~~~~~~~~~~~~~~~~~~~~~~~~~~~~~~~~~~~~~~~~~~~~~~~~~~~~~~~~~~~~~~~
% Tables (Tabular)
% ~~~~~~~~~~~~~~~~~~~~~~~~~~~~~~~~~~~~~~~~~~~~~~~~~~~~~~~~~~~~~~~~~~~~~~~~

\usepackage{booktabs}
\usepackage{ltxtable} % Longtable + tabularx

% ~~~~~~~~~~~~~~~~~~~~~~~~~~~~~~~~~~~~~~~~~~~~~~~~~~~~~~~~~~~~~~~~~~~~~~~~
% text related packages
% ~~~~~~~~~~~~~~~~~~~~~~~~~~~~~~~~~~~~~~~~~~~~~~~~~~~~~~~~~~~~~~~~~~~~~~~~

\usepackage{url}            % Befehl \url{...}
\usepackage{enumitem}		% Kompakte Listen

% Neue Befehle: \Centering, \RaggedLeft, and \RaggedRight, ... 
\usepackage{ragged2e}


% ~~~~~~~~~~~~~~~~~~~~~~~~~~~~~~~~~~~~~~~~~~~~~~~~~~~~~~~~~~~~~~~~~~~~~~~~
% Citations
% ~~~~~~~~~~~~~~~~~~~~~~~~~~~~~~~~~~~~~~~~~~~~~~~~~~~~~~~~~~~~~~~~~~~~~~~~

%\usepackage[
%	style=alphabetic, % Loads the bibliography and the citation style 
%	natbib=true, % define natbib compatible cite commands
%]{biblatex}	
% Other options:
%	style=numeric, % 
%	style=numeric-comp,    % [1–3, 7, 8]
%	style=numeric-verb,    % [2]; [5]; [6]


% ~~~~~~~~~~~~~~~~~~~~~~~~~~~~~~~~~~~~~~~~~~~~~~~~~~~~~~~~~~~~~~~~~~~~~~~~
% layout packages
% ~~~~~~~~~~~~~~~~~~~~~~~~~~~~~~~~~~~~~~~~~~~~~~~~~~~~~~~~~~~~~~~~~~~~~~~~
%
% Befehle für 1,5 und 2 zeilig: 
% \singlespacing, \onehalfspacing und \doublespacing
\usepackage{setspace}

% ~~~~~~~~~~~~~~~~~~~~~~~~~~~~~~~~~~~~~~~~~~~~~~~~~~~~~~~~~~~~~~~~~~~~~~~~
% Kopf und Fusszeile
% ~~~~~~~~~~~~~~~~~~~~~~~~~~~~~~~~~~~~~~~~~~~~~~~~~~~~~~~~~~~~~~~~~~~~~~~~

% Kopf und Fusszeile mit scrpage2 einstellen
\usepackage[automark, komastyle, nouppercase]{scrpage2}

% ~~~~~~~~~~~~~~~~~~~~~~~~~~~~~~~~~~~~~~~~~~~~~~~~~~~~~~~~~~~~~~~~~~~~~~~~
% pdf packages
% ~~~~~~~~~~~~~~~~~~~~~~~~~~~~~~~~~~~~~~~~~~~~~~~~~~~~~~~~~~~~~~~~~~~~~~~~

% Include pages from external PDF documents in LaTeX documents
\usepackage{pdfpages} 

% Optischer Randausgleich mit pdfTeX
\usepackage{microtype}

\usepackage[unicode]{hyperref}

\usepackage{listings}
% Einstellungen und Layoutstile
% ~~~~~~~~~~~~~~~~~~~~~~~~~~~~~~~~~~~~~~~~~~~~~~~~~~~~~~~~~~~~~~~~~~~~~~~~
% Colors
% ~~~~~~~~~~~~~~~~~~~~~~~~~~~~~~~~~~~~~~~~~~~~~~~~~~~~~~~~~~~~~~~~~~~~~~~~
\definecolor{sectioncolor}{RGB}{0, 0, 0}     % black

% ~~~~~~~~~~~~~~~~~~~~~~~~~~~~~~~~~~~~~~~~~~~~~~~~~~~~~~~~~~~~~~~~~~~~~~~~
% text related 
% ~~~~~~~~~~~~~~~~~~~~~~~~~~~~~~~~~~~~~~~~~~~~~~~~~~~~~~~~~~~~~~~~~~~~~~~~

%% style of URL
\urlstyle{tt}


% Keine hochgestellten Ziffern in der Fussnote (KOMA-Script-spezifisch):
\deffootnote{1.5em}{1em}{\makebox[1.5em][l]{\thefootnotemark}}

% Limit space of footnotes to 10 lines
\setlength{\dimen\footins}{10\baselineskip}

% prevent continuation of footnotes 
% at facing page
\interfootnotelinepenalty=10000 

% ~~~~~~~~~~~~~~~~~~~~~~~~~~~~~~~~~~~~~~~~~~~~~~~~~~~~~~~~~~~~~~~~~~~~~~~~
% Science
% ~~~~~~~~~~~~~~~~~~~~~~~~~~~~~~~~~~~~~~~~~~~~~~~~~~~~~~~~~~~~~~~~~~~~~~~~

\sisetup{%
	mode = math, detect-family, detect-weight,	
	exponent-product = \cdot,
	number-unit-separator=\text{\,},
	output-decimal-marker={,},
}

% ~~~~~~~~~~~~~~~~~~~~~~~~~~~~~~~~~~~~~~~~~~~~~~~~~~~~~~~~~~~~~~~~~~~~~~~~
% Citations / Style of Bibliography
% ~~~~~~~~~~~~~~~~~~~~~~~~~~~~~~~~~~~~~~~~~~~~~~~~~~~~~~~~~~~~~~~~~~~~~~~~

% Kommentar entfernene wenn biblatex geladen wird
% \IfPackageLoaded{biblatex}{%
	\ExecuteBibliographyOptions{%
%--- Backend --- --- ---
	backend=bibtex,  % (bibtex, bibtex8, biber)
	bibwarn=true, %
	bibencoding=ascii, % (ascii, inputenc, <encoding>)
%--- Sorting --- --- ---
	sorting=nty, % Sort by name, title, year.
	% other options: 
	% nty        Sort by name, title, year.
	% nyt        Sort by name, year, title.
	% nyvt       Sort by name, year, volume, title.
	% anyt       Sort by alphabetic label, name, year, title.
	% anyvt      Sort by alphabetic label, name, year, volume, title.
	% ynt        Sort by year, name, title.
	% ydnt       Sort by year (descending), name, title.
	% none       Do not sort at all. All entries are processed in citation order.
	% debug      Sort by entry key. This is intended for debugging only.
	%
	sortcase=true,
	sortlos=los, % (bib, los) The sorting order of the list of shorthands
	sortcites=false, % do/do not sort citations according to bib	
%--- Dates --- --- ---
	date=comp,  % (short, long, terse, comp, iso8601)
%	origdate=
%	eventdate=
%	urldate=
%	alldates=
	datezeros=true, %
	dateabbrev=true, %
%--- General Options --- --- ---
	maxnames=1,
	minnames=1,
%	maxbibnames=99,
%	maxcitenames=1,
%	autocite= % (plain, inline, footnote, superscript) 
	autopunct=true,
	language=auto,
	babel=none, % (none, hyphen, other, other*)
	block=none, % (none, space, par, nbpar, ragged)
	notetype=foot+end, % (foot+end, footonly, endonly)
	hyperref=true, % (true, false, auto)
	backref=true,
	backrefstyle=three, % (none, three, two, two+, three+, all+)
	backrefsetstyle=setonly, %
	indexing=false, % 
	% options:
	% true       Enable indexing globally.
	% false      Disable indexing globally.
	% cite       Enable indexing in citations only.
	% bib        Enable indexing in the bibliography only.
	refsection=none, % (part, chapter, section, subsection)
	refsegment=none, % (none, part, chapter, section, subsection)
	abbreviate=true, % (true, false)
	defernumbers=false, % 
	punctfont=false, % 
	arxiv=abs, % (ps, pdf, format)	
%--- Style Options --- --- ---	
% The following options are provided by the standard styles
	isbn=false,%
	url=false,%
	doi=false,%
	eprint=false,%	
	}%	
	
	% change alpha label to be without +	
	\renewcommand*{\labelalphaothers}{}
	
	% change 'In: <magazine>" to "<magazine>"
	\renewcommand*{\intitlepunct}{}
	\DefineBibliographyStrings{german}{in={}}
	
	% make names capitalized \textsc{}
	\renewcommand{\mkbibnamefirst}{\textsc}
	\renewcommand{\mkbibnamelast}{\textsc}
	
	% make volume and number look like 
	% 'Bd. 33(14): '
	\renewbibmacro*{volume+number+eid}{%
	  \setunit{\addcomma\space}%
	  \bibstring{volume}% 
	  \setunit{\addspace}%
	  \printfield{volume}%
	  \iffieldundef{number}{}{% 
	    \printtext[parens]{%
	      \printfield{number}%
	    }%
	  }%
	  \setunit{\addcomma\space}%
	  \printfield{eid}
	  %\setunit{\addcolon\space}%
	  }	

	% <authors>: <title>
	\renewcommand*{\labelnamepunct}{\addcolon\space}
	% make ': ' before pages
	\renewcommand*{\bibpagespunct}{\addcolon\space}
	% names delimiter ';' instead of ','
	%\renewcommand*{\multinamedelim}{\addsemicolon\space}

	% move date before issue
	\renewbibmacro*{journal+issuetitle}{%
	  \usebibmacro{journal}%
	  \setunit*{\addspace}%
	  \iffieldundef{series}
	    {}
	    {\newunit
	     \printfield{series}%
	     \setunit{\addspace}}%
	  %
	  \usebibmacro{issue+date}%
	  \setunit{\addcolon\space}%
	  \usebibmacro{issue}%
	  \setunit{\addspace}%
	  \usebibmacro{volume+number+eid}%
	  \newunit}

	% print all names, even if maxnames = 1
	\DeclareCiteCommand{\citeauthors}
	  {
	   \defcounter{maxnames}{1000}
	   \boolfalse{citetracker}%
	   \boolfalse{pagetracker}%
	   \usebibmacro{prenote}}
	  {\ifciteindex
	     {\indexnames{labelname}}
	     {}%
	   \printnames{labelname}}
	  {\multicitedelim}
	  {\usebibmacro{postnote}}

}%

% ~~~~~~~~~~~~~~~~~~~~~~~~~~~~~~~~~~~~~~~~~~~~~~~~~~~~~~~~~~~~~~~~~~~~~~~~
% figures, placement, floats and captions
% ~~~~~~~~~~~~~~~~~~~~~~~~~~~~~~~~~~~~~~~~~~~~~~~~~~~~~~~~~~~~~~~~~~~~~~~~

% Make float placement easier
\renewcommand{\floatpagefraction}{.75} % vorher: .5
\renewcommand{\textfraction}{.1}       % vorher: .2
\renewcommand{\topfraction}{.8}        % vorher: .7
\renewcommand{\bottomfraction}{.5}     % vorher: .3
\setcounter{topnumber}{3}        % vorher: 2
\setcounter{bottomnumber}{2}     % vorher: 1
\setcounter{totalnumber}{5}      % vorher: 3

%% ~~~ Captions ~~~~~~~~~~~~~~~~~~~~~~~~~~~~~~~~~~~~~~~~~~~~~~~~~~~~~~~~~~
% Style of captions
\DeclareCaptionStyle{captionStyleTemplateDefault}
[ % single line captions
   justification = centering
]
{ % multiline captions
% -- Formatting
   format      = plain,  % plain, hang
   indention   = 0em,    % indention of text 
   labelformat = default,% default, empty, simple, brace, parens
   labelsep    = colon,  % none, colon, period, space, quad, newline, endash
   textformat  = simple, % simple, period
% -- Justification
   justification = justified, %RaggedRight, justified, centering
   singlelinecheck = true, % false (true=ignore justification setting in single line)
% -- Fonts
   labelfont   = {small,bf},
   textfont    = {small,rm},
% valid values:
% scriptsize, footnotesize, small, normalsize, large, Large
% normalfont, ip, it, sl, sc, md, bf, rm, sf, tt
% singlespacing, onehalfspacing, doublespacing
% normalcolor, color=<...>
%
% -- Margins and further paragraph options
   margin = 10pt, %.1\textwidth,
   % width=.8\linewidth,
% -- Skips
   skip     = 10pt, % vertical space between the caption and the figure
   position = auto, % top, auto, bottom
% -- Lists
   % list=no, % suppress any entry to list of figure 
   listformat = subsimple, % empty, simple, parens, subsimple, subparens
% -- Names & Numbering
   % figurename = Abb. %
   % tablename  = Tab. %
   % listfigurename=
   % listtablename=
   % figurewithin=chapter
   % tablewithin=chapter
%-- hyperref related options
	hypcap=true, % (true, false) 
	% true=all hyperlink anchors are placed at the 
	% beginning of the (floating) environment
	%
	hypcapspace=0.5\baselineskip
}

% apply caption style
\captionsetup{
	style = captionStyleTemplateDefault % base
}

% Predefinded skip setup for different floats
\captionsetup[table]{position=top}
\captionsetup[figure]{position=bottom}


% options for subcaptions
\captionsetup[sub]{ %
	style = captionStyleTemplateDefault, % base
	skip=6pt,
	margin=5pt,
	labelformat = parens,% default, empty, simple, brace
	labelsep    = space,
	list=false,
	hypcap=false
}

% ~~~~~~~~~~~~~~~~~~~~~~~~~~~~~~~~~~~~~~~~~~~~~~~~~~~~~~~~~~~~~~~~~~~~~~~~
% layout 
% ~~~~~~~~~~~~~~~~~~~~~~~~~~~~~~~~~~~~~~~~~~~~~~~~~~~~~~~~~~~~~~~~~~~~~~~~


%% Paragraph Separation =================================
\KOMAoptions{%
   parskip=absolute, % do not change indentation according to fontsize
   parskip=false     % indentation of 1em
   % parskip=half    % parksip of 1/2 line 
}%

%% line spacing =========================================
%\onehalfspacing	% 1,5-facher Abstand
%\doublespacing		% 2-facher Abstand

%% page layout ==========================================

\raggedbottom     % Variable Seitenhoehen zulassen

% Koma Script text area layout
\KOMAoptions{%
   DIV=11,% (Size of Text Body, higher values = greater textbody)
   BCOR=5mm% (Bindekorrektur)
}%

%%% === Page Layout  Options ===
\KOMAoptions{% (most options are for package typearea)
   % twoside=true, % two side layout (alternating margins, standard in books)
   twoside=false, % single side layout 
   %
   headlines=2.1,%
}%

%\KOMAoptions{%
%      headings=noappendixprefix % chapter in appendix as in body text
%      ,headings=nochapterprefix  % no prefix at chapters
%      % ,headings=appendixprefix   % inverse of 'noappendixprefix'
%      % ,headings=chapterprefix    % inverse of 'nochapterprefix'
%      % ,headings=openany   % Chapters start at any side
%      % ,headings=openleft  % Chapters start at left side
%      ,headings=openright % Chapters start at right side      
%}%


% reloading of typearea, necessary if setting of spacing changed
\typearea[current]{last}

% ~~~~~~~~~~~~~~~~~~~~~~~~~~~~~~~~~~~~~~~~~~~~~~~~~~~~~~~~~~~~~~~~~~~~~~~~
% Titlepage
% ~~~~~~~~~~~~~~~~~~~~~~~~~~~~~~~~~~~~~~~~~~~~~~~~~~~~~~~~~~~~~~~~~~~~~~~~
\KOMAoptions{%
   % titlepage=true %
   titlepage=false %
}%

% ~~~~~~~~~~~~~~~~~~~~~~~~~~~~~~~~~~~~~~~~~~~~~~~~~~~~~~~~~~~~~~~~~~~~~~~~
% head and foot lines
% ~~~~~~~~~~~~~~~~~~~~~~~~~~~~~~~~~~~~~~~~~~~~~~~~~~~~~~~~~~~~~~~~~~~~~~~~

% \pagestyle{scrheadings} % Seite mit Headern
\pagestyle{scrplain} % Seiten ohne Header

% loescht voreingestellte Stile
\clearscrheadings
\clearscrplain
%
% Was steht wo...
% Bei headings:
%   % Oben aussen: Kapitel und Section
%   % Unten aussen: Seitenzahl
%   \ohead{\pagemark}
%   \ihead{\headmark}
%   \ofoot[\pagemark]{} % Außen unten: Seitenzahlen bei plain
% Bei Plain:
\cfoot[\pagemark]{\pagemark} % Mitte unten: Seitenzahlen bei plain


% Angezeigte Abschnitte im Header
% \automark[section]{chapter} %[rechts]{links}
\automark[subsection]{section} %[rechts]{links}

% ~~~~~~~~~~~~~~~~~~~~~~~~~~~~~~~~~~~~~~~~~~~~~~~~~~~~~~~~~~~~~~~~~~~~~~~~
% headings / page opening
% ~~~~~~~~~~~~~~~~~~~~~~~~~~~~~~~~~~~~~~~~~~~~~~~~~~~~~~~~~~~~~~~~~~~~~~~~
\setcounter{secnumdepth}{2}

\KOMAoptions{%
%%%% headings
   % headings=small  % Small Font Size, thin spacing above and below
   % headings=normal % Medium Font Size, medium spacing above and below
   headings=big % Big Font Size, large spacing above and below
}%

% Titelzeile linksbuendig, haengend
\renewcommand*{\raggedsection}{\raggedright} 

% ~~~~~~~~~~~~~~~~~~~~~~~~~~~~~~~~~~~~~~~~~~~~~~~~~~~~~~~~~~~~~~~~~~~~~~~~
% fonts of headings
% ~~~~~~~~~~~~~~~~~~~~~~~~~~~~~~~~~~~~~~~~~~~~~~~~~~~~~~~~~~~~~~~~~~~~~~~~
\setkomafont{sectioning}{\normalfont\sffamily} % \rmfamily
\setkomafont{descriptionlabel}{\itshape}
\setkomafont{pageheadfoot}{\normalfont\normalcolor\small\sffamily}
\setkomafont{pagenumber}{\normalfont\sffamily}

%%% --- Titlepage ---
%\setkomafont{subject}{}
%\setkomafont{subtitle}{}
%\setkomafont{title}{}

% ~~~~~~~~~~~~~~~~~~~~~~~~~~~~~~~~~~~~~~~~~~~~~~~~~~~~~~~~~~~~~~~~~~~~~~~~
% settings and layout of TOC, LOF, 
% ~~~~~~~~~~~~~~~~~~~~~~~~~~~~~~~~~~~~~~~~~~~~~~~~~~~~~~~~~~~~~~~~~~~~~~~~
\setcounter{tocdepth}{3} % Depth of TOC Display

% ~~~~~~~~~~~~~~~~~~~~~~~~~~~~~~~~~~~~~~~~~~~~~~~~~~~~~~~~~~~~~~~~~~~~~~~~
% Tabellen
% ~~~~~~~~~~~~~~~~~~~~~~~~~~~~~~~~~~~~~~~~~~~~~~~~~~~~~~~~~~~~~~~~~~~~~~~~

%%% -| Neue Spaltendefinitionen 'columntypes' |--
%
% Belegte Spaltentypen:
% l - links
% c - zentriert
% r - rechts
% p,m,b  - oben, mittig, unten
% X - tabularx Auto-Spalte

% um Tabellenspalten mit Flattersatz zu setzen, muss \\ vor
% (z.B.) \raggedright geschuetzt werden:
\newcommand{\PreserveBackslash}[1]{\let\temp=\\#1\let\\=\temp}

% Spalten mit Flattersatz und definierte Breite:
% m{} -> mittig
% p{} -> oben
% b{} -> unten
%
% Linksbuendig:
\newcolumntype{v}[1]{>{\PreserveBackslash\RaggedRight\hspace{0pt}}p{#1}}
\newcolumntype{M}[1]{>{\PreserveBackslash\RaggedRight\hspace{0pt}}m{#1}}
% % Rechtsbuendig :
% \newcolumntype{R}[1]{>{\PreserveBackslash\RaggedLeft\hspace{0pt}}m{#1}}
% \newcolumntype{S}[1]{>{\PreserveBackslash\RaggedLeft\hspace{0pt}}p{#1}}
% % Zentriert :
% \newcolumntype{Z}[1]{>{\PreserveBackslash\Centering\hspace{0pt}}m{#1}}
% \newcolumntype{A}[1]{>{\PreserveBackslash\Centering\hspace{0pt}}p{#1}}

\newcolumntype{Y}{>{\PreserveBackslash\RaggedLeft\hspace{0pt}}X}

%-- Einstellungen für Tabellen ----------
\providecommand\tablestyle{%
  \renewcommand{\arraystretch}{1.4} % Groessere Abstaende zwischen Zeilen
  \normalfont\normalsize            %
  \sffamily\small           % Serifenlose und kleine Schrift
  \centering%                       % Tabelle zentrieren
}

%--Einstellungen für Tabellen ----------

\colorlet{tablesubheadcolor}{gray!40}
\colorlet{tableheadcolor}{gray!25}
\colorlet{tableblackheadcolor}{black!60}
\colorlet{tablerowcolor}{gray!15.0}


% ~~~~~~~~~~~~~~~~~~~~~~~~~~~~~~~~~~~~~~~~~~~~~~~~~~~~~~~~~~~~~~~~~~~~~~~~
% pdf packages
% ~~~~~~~~~~~~~~~~~~~~~~~~~~~~~~~~~~~~~~~~~~~~~~~~~~~~~~~~~~~~~~~~~~~~~~~~

% ~~~~~~~~~~~~~~~~~~~~~~~~~~~~~~~~~~~~~~~~~~~~~~~~~~~~~~~~~~~~~~~~~~~~~~~~
% fix remaining problems
% ~~~~~~~~~~~~~~~~~~~~~~~~~~~~~~~~~~~~~~~~~~~~~~~~~~~~~~~~~~~~~~~~~~~~~~~~




% ~~~~~~~~~~~~~~~~~~~~~~~~~~~~~~~~~~~~~~~~~~~~~~~~~~~~~~~~~~~~~~~~~~~~~~~~
% Eigene Befehle
% ~~~~~~~~~~~~~~~~~~~~~~~~~~~~~~~~~~~~~~~~~~~~~~~~~~~~~~~~~~~~~~~~~~~~~~~~
% -- new commands --
\providecommand{\abs}[1]{\lvert#1\rvert}
\providecommand{\Abs}[1]{\left\lvert#1\right\rvert}
\providecommand{\norm}[1]{\left\Vert#1\right\Vert}
\providecommand{\Trace}[1]{\ensuremath{\Tr\{\,#1\,\}}} % Trace /Spur
%

\renewcommand{\d}{\partial\mspace{2mu}} % partial diff
\newcommand{\td}{\,\mathrm{d}}	% total diff

\newcommand{\Ham}{\mathcal{H}}    % Hamilton
\newcommand{\Prob}{\mathscr{P}}    % Hamilton
\newcommand{\unity}{\mathds{1}}   % Real

\renewcommand{\i}{\mathrm{i}}   % imagin�re Einheit



% -- New Operators --
\DeclareMathOperator{\rot}{rot}
\DeclareMathOperator{\grad}{grad}
\DeclareMathOperator{\Tr}{Tr}
\DeclareMathOperator{\const}{const}
\DeclareMathOperator{\e}{e} 			% exponatial Function

\newcommand{\command}[1]{\texttt{\lstinline$#1$}}


% ~~~~~~~~~~~~~~~~~~~~~~~~~~~~~~~~~~~~~~~~~~~~~~~~~~~~~~~~~~~~~~~~~~~~~~~~
% Eigene Befehle
% ~~~~~~~~~~~~~~~~~~~~~~~~~~~~~~~~~~~~~~~~~~~~~~~~~~~~~~~~~~~~~~~~~~~~~~~~
% Silbentrennung hinzufügen als 
% Sil-ben-tren-nung 
\hyphenation{}

\listfiles % schreibt alle verwendeten Dateien in die log Datei

%% Dokument Beginn %%%%%%%%%%%%%%%%%%%%%%%%%%%%%%%%%%%%%%%%%%%%%%%%%%%%%%%%
\begin{document}

% Automatische Titelseite

%\subject{Praktikumsprotokoll}
%\title{ Bash \\ \normalsize Praktikum  1}
%\author{Aljoscha Pörtner \& Max Mustermann}
%\date{09.02.2015}
%\maketitle

% Manuelle Titelseite

\begin{titlepage}
   \mbox{}\vspace{5\baselineskip}\\
   \sffamily\huge
   \centering
   % Titel
   {\Huge Betriebssysteme} \\
  
 Multi-Threading und Synchronisation\\ \normalsize Praktikum 6
   \vspace{3\baselineskip}\\
   \rmfamily\Large
  Fachhochschule Bielefeld \\
  Campus Minden \\
  Studiengang Informatik
   \vspace{2\baselineskip}\\
\noindent\rule{15cm}{0.4pt}
Beteiligte Personen:
\begin{table}[H]
	\tablestyle
	\rowcolors{1}{tablerowcolor}{white!100}
	\begin{tabular}{*{2}{v{0.45\textwidth}}}
		\hline
		\rowcolor{tableheadcolor}
		\textbf{Name} &
		\textbf{Matrikelnummer} \tabularnewline
		\hline
		%
		Mirko Weidemann Kreitz & 1048290 \tabularnewline
		Oxana Zhurakovskaya  & 130157 \tabularnewline
		Karsten Michael Tymann & 1047529 \tabularnewline
		Yuliia Dobranska & 1093568\tabularnewline

	\end{tabular}
\end{table}

   \noindent\rule{15cm}{0.4pt}
      \vspace{1\baselineskip}\\
   \today
\end{titlepage}


\tableofcontents
\newpage
% Testdokumente (auskommentieren!)
%\input{content/hinweis.tex}
%\input{content/demo/demo.tex}
%\input{content/demo/latexexample.tex}

% in diese Datei gehört der Inhalt des Dokumentes:
\section{Aufgabe 1}
Sie sollen in dieser Aufgabe ein Programm entwickeln, das mehrere einzelne Dateien einliest,
komprimiert und als neue Datei speichert.
\begin{itemize}
	\item Dem Programm wird per Kommandozeile ein Ordnerpfad übergeben.
	\item Arbeiten Sie mit mehreren Threads. Ein Thread (Leser-Thread) liest die Dateien in dem
	übergebenen Ordner ein und hängt den Inhalt sowie den Dateipfad an eine Queue an.
	Nutzen Sie dafür eine \command{struct Job}.
	\item Eine zur Compilezeit konfigurierbare Anzahl von Threads (Komprimierungs-Thread) liest
	jeweils einen Job aus der Queue, komprimiert seinen Inhalt und speichert diesen
	im Format <alter Dateiname>.compr. Dies soll solange wiederholt werden,
	bis die Queue leer und der Leser-Thread beendet ist.
	\item Der Leser-Thread soll Dateien, die mit .compr enden, ignorieren.
	\item Ein Komprimierungs-Thread bekommt bei seiner Erstellung als Parameter eine
	Instanz-nummer zugeordnet, die ihn z.B. bei Debug-Ausgaben eindeutig identifiziert.
	\item Die Zugriffe auf die Queue müssen synchronisiert, d.h. gegeneinander geschützt sein.
	\item  Fügen Sie im Leser-Thread nach dem Einlesen einer Datei ein \command{sleep(1)} und in den
	Kompressions-Threads nach dem Komprimieren ein \command{sleep(3)}  ein, um eine langsame Festplatte
	und einen komplexen Kompressionsalgorithmus zu simulieren. Beobachten Sie, wie ihr Programm mit
	und ohne die \command{sleep}-Anweisungen arbeitet.
	\item Bestimmen Sie n ̈aherungsweise die Laufzeit bei unterschiedlicher Anzahl von
	Kompressions-Threads, z. B. mit der Funktion \command{sdifftime()} in \command{main()}.

\end{itemize}
	\subsection{Vorbereitung}
	Als vorbereitung haben wir die Datei \textit{main.c} erstellt, wo unsere program implemetiert wird,
	und die empfohlenen Headers(\textit{queue.h}, \textit{miniz.h},\textit{pthread.h}) includiert.
	\subsection{Durchführung}
\begin{itemize}
	\item Damit an programm als Parameter ein Pfad zu den ordner übergegeben werden kann, haben wir die main
	Methode so geschrieben, dass sie Argumente annehmen kann:
\begin{lstlisting}{language=C}
int main(int argc, char *argv[]){
	...
	return 0;
}
\end{lstlisting}
	\command{int argc} beinhaltet immer die Anzahl der übergegebene Parameter,
	\command{char *argv[]} beinhaltet die Parameter, die an Programm übergegeben wurde,
	dabei \command{argv[0]} beinhaltet immer den Namen des Programms.
	Ab \command{argv[1]} sind die von Benutzer angegebenen Parameter zugreifbar.
	Da an unsere Programm nur ein Argument übergegeben werden soll,
	beachten wir nur die Inhalt von \command{argv[1]}.
	\item wir definieren constante länge für die Pfad zu den Ordner mit \command{#define MAX_PATH 1024}
	\item wir definieren constante Anzahl der Threads \command{#define COMPILETHREADS 10}
	\item wir definieren Variable \command{pthread_mutex_t lock;}, die genutzt wird,
	um Thread zu synchronisieren.
	\item um eine Job darzustellen definieren wir eine Struktur job mit typedef Job,
	die eine Pfad zu dem Ordner beinhalten wird und Name der zukomprimierenden Datei.
	\item die Strucktur beinhatetden:
	\begin{itemize}
		\item Pfad(\command{char *path}) zu den Ordner (der Pfad ist absolute Pfad zu dem Ordner, der als argument an main
	übergegeben wurde)
	\item Filename (\command{char *filename}), die den absoluten pfad zu den zu komprimierenden Datei enthält
	\item neue Filename \command{char *newfilename}), die den absoluten pfad zu den zu der resultierende(d.h komprimierte) Datei enthält
\end{itemize}
\begin{lstlisting}
typedef struct job{
	char *path;
	char *filename;
	char *newfilename;
}Job;
\end{lstlisting}
	\item folglich wir definieren Variable \command{static Queue jobQueue;}, die JobQueue darstellt
	\item in \command{main} inizialisieren wir \command{jobQueue = queue_create();}
	\item für den Leser-Thread implementieren wir zuerst die Funktion \command{void *readPath(void *path)},
	die als Paremeter einen char pointer erhaten soll(pfad die der main methode übergegeben wurde),
	wird aber wegen pthread nutzung als void pointer deklariert.
	\begin{itemize}
		\item wir definieren Variable \command{char resolved_path[MAX_PATH]},
		die eine resolved Pfad beinhalten wird.
		\item um Ordner information zu lesen nutzen wir Bibliothek \command{dirent.h},
		außerdem definieren wir eine Variable  \command{DIR *dir = NULL;}, die directory stream beinhalten wird.
		\item um die Ordner inhalt aus dem directory stream zu lesen,definieren wir eine Variable
		\command{struct dirent *dptr = NULL;}
		\item da jeder Ordner Dateien \textit{.}und \textit{..} beinhaltet, die ...
		definieren wir zwei Variablen \command{char *dot = \".";} und \command{char *dotdot = \"..";},
		die wir zum vergleich nutzen werden
		\item um die schon komprimierte dateien zu ignorieren, erstellen wir eineVariable \comand{char * compressed=``.compr;''}
		die wir später zum vergleich der Dateinamen nutzen
		\item um absolute pfad zu erhalten nutzen wir die Funktion \command{realpath((char*)pfad, resolved_path)} aus der Standartbibliothek,
		als parameter übergeben wir die von Benutzer erhaltene Pfad \command{path}
		und unsere Variable \command{resolved_path}, die den ergebnis erhalten wird,
		dabei casten wir Variable \comand{path} zu char pointer.
		die Funktion wird ein \command{NULL} pointer zurückliefern, falls beim Pfadname auflösun ein Fehler auftritt,
		deswegen können wir den Ergebnis der \command{realpath} in der If-Abfrage überprüfen,
		und fall etwas mit dem Pfad nicht stimmt, kann die Funktion \command{readPath} ihr Arbeit abbrechen.
	\begin{lstlisting}
if(realpath((char*)path, resolved_path)){}
	\end{lstlisting}
		\item wenn Pfad erfolgreich aufgelöst wurde können wir weiter vorgehen.
		\item wir öffnen directory stream mit der Funktion \command{opendir(resolved_path)},
		die Funktion wird \command{NULL} pointer zurückliefern wenn ein Fehler beim öffnen auftritt.
		Wir fragen den resultet ebenfalls in der If-Abfrage, wenn der Ordner erfolgreich geöffnet wurde,
		kann man weiter vorgehen, sons muss die Funktion \command{readPath} ihr Arbeit abbrechen.
\begin{lstlisting}
if((dir = opendir(resolved_path))){}
	\end{lstlisting}
		\item nun kann man mit der While-Schleife durch einzelne Einträge im directory stream durch iterieren,
		dabei hilft uns die Variable \command{dptr}, die bei jeder Iteration auf nächste Eintrag zeigt.
		Um nächsten Eintrag auszulesen, benutzen wir Funktion  \command{readdir(dir)},
		die als Parameter directory stream annimmt
\begin{lstlisting}
while((dptr = readdir(dir))){}
\end{lstlisting}
		\item wir schließen die Dateien mit Namen \textit{.}, \textit{..} und mit endung \textit{.compr}
		von Komprimieren aus,
		indem wir Name des Aktuellen eintrags mit Variablen \command{dot} und \command{dotdot} vergleichen.
		Und wenn es sich um diese Dateien handelt, geht die while schleife
		ohne weiteres vorgenhen zum nächsten Iteration.
\begin{lstlisting}
if(strcmp(dptr->d_name, dot) == 0 ||
	strcmp(dptr->d_name, dotdot)==0 ||
	strstr(dptr->d_name, compressed) != NULL){
	continue;
}
\end{lstlisting}
	\item folglich wir allosieren Speicherplatz für ein Job und Prüfen ob der vorgang geklappt hat,
	wenn nicht, dann bricht das programm ab
\begin{lstlisting}
	Job *j = (Job *) malloc(sizeof(Job));
	if (!j) {
		exit(EXIT_FAILURE);
	}
\end{lstlisting}
	\item um den neuen Job zu inizialisieren, haben wir eine Funktion erstellt
	\command{void setJob(char *path, char *file, Job* job)}.
	 \begin{itemize}
		 \item die Funktion nimmt als Argumente pointer auf pfad zu dem ordner, pointer auf filename
	 (die von den directory stream gelesen  wurde) und den pointer auf Job struktur,
	 die inizialisiert werden soll, an.
	 \item Für die komprimierte Datei brauchen wir die Endung, die wir in Variable speichern:
	 \command{char * compressed = ``.compr'';}
	 \item Wir allosieren für die Job-Elemente \command{path}, \command{filename} und \command{newfilename}
	 neue speicherplatz
	 \item in den Feld \command{path} copieren wir den übergegeben Path zu dem Ordner
	 \command{strcpy(job->path, path);}
	 \item in den Feld \command{filename} kopieren wir den übergegeben Path zu dem Ordner
	 \command{strcpy(job->filename, job->path);}, und hängen noch \command{\} und \command{filename} an,
	 damit die Variable absoluten Pfad zu dateiname enthält \command{strcat(job->filename,"/");} und
	 \command{strcat(job->filename,file);}
	 \item in den Feld \command{filename} kopieren wir den pfad zu dateiname
	 \command{strcpy(job->newfilename,job->filename);} und hängen die Endung  an \command{strcat(job->newfilename,compressed);}
 \end{itemize}
 \begin{lstlisting}
	 void setJob(char *path, char *file, Job* job){
 	printf("setJob----- \n\n");
 	char * compressed = ".compr";
 	printf("path: %s\n",path);
 	printf("filename: %s\n",file);
 	printf("job path: %s\n",job->path);

 	job->path = (char *) malloc(MAX_PATH);
 	strcpy(job->path, path);

 	job->filename = (char *) malloc(MAX_PATH);
 	strcpy(job->filename,job->path);

 	strcat(job->filename,"/");
 	strcat(job->filename,file);

 	job->newfilename = (char *) malloc(MAX_PATH);
 	strcpy(job->newfilename,job->filename);
 	strcat(job->newfilename,compressed);

 	printf("job path: %s\n",job->path);
 	printf("job filename: %s\n",job->filename);
 	printf("job newfilename: %s\n\n",job->newfilename);
 }
 \end{lstlisting}
	\item anschließend fügen wir neue Job zu der jobQueue hinzu
	\begin{lstlisting
	queue_insert(jobQueue, j);
\end{lstlisting}
	\item nach vollständige Iteration, schließen wir directory stream \command{closedir(dir);}
\end{itemize}
\end{itemize}
	\subsection{Fazit}

\newpage


\end{document}
%% Dokument ENDE %%%%%%%%%%%%%%%%%%%%%%%%%%%%%%%%%%%%%%%%%%%%%%%%%%%%%%%%%%
