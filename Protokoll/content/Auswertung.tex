\section{Aufgabe 1}
Sie sollen in dieser Aufgabe ein Programm entwickeln, das mehrere einzelne Dateien einliest,
komprimiert und als neue Datei speichert.
\begin{itemize}
	\item Dem Programm wird per Kommandozeile ein Ordnerpfad übergeben.
	\item Arbeiten Sie mit mehreren Threads. Ein Thread (Leser-Thread) liest die Dateien in dem
	übergebenen Ordner ein und h ̈angt den Inhalt sowie den Dateipfad an eine Queue an.
	Nutzen Sie dafür eine \command{struct Job}.
	\item Eine zur Compilezeit konfigurierbare Anzahl von Threads (Komprimierungs-Thread) liest
	jeweils einen Job aus der Queue, komprimiert seinen Inhalt und speichert diesen
	im Format <alter Dateiname>.compr. Dies soll solange wiederholt werden,
	bis die Queue leer und der Leser-Thread beendet ist.
	\item Der Leser-Thread soll Dateien, die mit .compr enden, ignorieren.
	\item Ein Komprimierungs-Thread bekommt bei seiner Erstellung als Parameter eine
	Instanz-nummer zugeordnet, die ihn z.B. bei Debug-Ausgaben eindeutig identifiziert.
	\item Die Zugriffe auf die Queue müssen synchronisiert, d.h. gegeneinander geschützt sein.
	\item  Fügen Sie im Leser-Thread nach dem Einlesen einer Datei ein \command{sleep(1)} und in den
	Kompressions-Threads nach dem Komprimieren ein \command{sleep(3)}  ein, um eine langsame Festplatte
	und einen komplexen Kompressionsalgorithmus zu simulieren. Beobachten Sie, wie ihr Programm mit
	und ohne die \command{sleep}-Anweisungen arbeitet.
	\item Bestimmen Sie n ̈aherungsweise die Laufzeit bei unterschiedlicher Anzahl von
	Kompressions-Threads, z. B. mit der Funktion \command{sdifftime()} in \command{main()}.

\end{itemize}
	\subsection{Vorbereitung}
	Als vorbereitung haben wir die Datei \textit{main.c} erstellt, wo unsere program implemetiert wird,
	und die empfohlenen Headers(\textit{queue.h}, \textit{miniz.h},\textit{pthread.h}) includiert.
	\subsection{Durchführung}

	\subsection{Fazit}

\newpage
